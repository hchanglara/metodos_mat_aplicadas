% Options for packages loaded elsewhere
\PassOptionsToPackage{unicode}{hyperref}
\PassOptionsToPackage{hyphens}{url}
%
\documentclass[
]{book}
\usepackage{amsmath,amssymb}
\usepackage{lmodern}
\usepackage{iftex}
\ifPDFTeX
  \usepackage[T1]{fontenc}
  \usepackage[utf8]{inputenc}
  \usepackage{textcomp} % provide euro and other symbols
\else % if luatex or xetex
  \usepackage{unicode-math}
  \defaultfontfeatures{Scale=MatchLowercase}
  \defaultfontfeatures[\rmfamily]{Ligatures=TeX,Scale=1}
\fi
% Use upquote if available, for straight quotes in verbatim environments
\IfFileExists{upquote.sty}{\usepackage{upquote}}{}
\IfFileExists{microtype.sty}{% use microtype if available
  \usepackage[]{microtype}
  \UseMicrotypeSet[protrusion]{basicmath} % disable protrusion for tt fonts
}{}
\makeatletter
\@ifundefined{KOMAClassName}{% if non-KOMA class
  \IfFileExists{parskip.sty}{%
    \usepackage{parskip}
  }{% else
    \setlength{\parindent}{0pt}
    \setlength{\parskip}{6pt plus 2pt minus 1pt}}
}{% if KOMA class
  \KOMAoptions{parskip=half}}
\makeatother
\usepackage{xcolor}
\usepackage{color}
\usepackage{fancyvrb}
\newcommand{\VerbBar}{|}
\newcommand{\VERB}{\Verb[commandchars=\\\{\}]}
\DefineVerbatimEnvironment{Highlighting}{Verbatim}{commandchars=\\\{\}}
% Add ',fontsize=\small' for more characters per line
\usepackage{framed}
\definecolor{shadecolor}{RGB}{248,248,248}
\newenvironment{Shaded}{\begin{snugshade}}{\end{snugshade}}
\newcommand{\AlertTok}[1]{\textcolor[rgb]{0.94,0.16,0.16}{#1}}
\newcommand{\AnnotationTok}[1]{\textcolor[rgb]{0.56,0.35,0.01}{\textbf{\textit{#1}}}}
\newcommand{\AttributeTok}[1]{\textcolor[rgb]{0.77,0.63,0.00}{#1}}
\newcommand{\BaseNTok}[1]{\textcolor[rgb]{0.00,0.00,0.81}{#1}}
\newcommand{\BuiltInTok}[1]{#1}
\newcommand{\CharTok}[1]{\textcolor[rgb]{0.31,0.60,0.02}{#1}}
\newcommand{\CommentTok}[1]{\textcolor[rgb]{0.56,0.35,0.01}{\textit{#1}}}
\newcommand{\CommentVarTok}[1]{\textcolor[rgb]{0.56,0.35,0.01}{\textbf{\textit{#1}}}}
\newcommand{\ConstantTok}[1]{\textcolor[rgb]{0.00,0.00,0.00}{#1}}
\newcommand{\ControlFlowTok}[1]{\textcolor[rgb]{0.13,0.29,0.53}{\textbf{#1}}}
\newcommand{\DataTypeTok}[1]{\textcolor[rgb]{0.13,0.29,0.53}{#1}}
\newcommand{\DecValTok}[1]{\textcolor[rgb]{0.00,0.00,0.81}{#1}}
\newcommand{\DocumentationTok}[1]{\textcolor[rgb]{0.56,0.35,0.01}{\textbf{\textit{#1}}}}
\newcommand{\ErrorTok}[1]{\textcolor[rgb]{0.64,0.00,0.00}{\textbf{#1}}}
\newcommand{\ExtensionTok}[1]{#1}
\newcommand{\FloatTok}[1]{\textcolor[rgb]{0.00,0.00,0.81}{#1}}
\newcommand{\FunctionTok}[1]{\textcolor[rgb]{0.00,0.00,0.00}{#1}}
\newcommand{\ImportTok}[1]{#1}
\newcommand{\InformationTok}[1]{\textcolor[rgb]{0.56,0.35,0.01}{\textbf{\textit{#1}}}}
\newcommand{\KeywordTok}[1]{\textcolor[rgb]{0.13,0.29,0.53}{\textbf{#1}}}
\newcommand{\NormalTok}[1]{#1}
\newcommand{\OperatorTok}[1]{\textcolor[rgb]{0.81,0.36,0.00}{\textbf{#1}}}
\newcommand{\OtherTok}[1]{\textcolor[rgb]{0.56,0.35,0.01}{#1}}
\newcommand{\PreprocessorTok}[1]{\textcolor[rgb]{0.56,0.35,0.01}{\textit{#1}}}
\newcommand{\RegionMarkerTok}[1]{#1}
\newcommand{\SpecialCharTok}[1]{\textcolor[rgb]{0.00,0.00,0.00}{#1}}
\newcommand{\SpecialStringTok}[1]{\textcolor[rgb]{0.31,0.60,0.02}{#1}}
\newcommand{\StringTok}[1]{\textcolor[rgb]{0.31,0.60,0.02}{#1}}
\newcommand{\VariableTok}[1]{\textcolor[rgb]{0.00,0.00,0.00}{#1}}
\newcommand{\VerbatimStringTok}[1]{\textcolor[rgb]{0.31,0.60,0.02}{#1}}
\newcommand{\WarningTok}[1]{\textcolor[rgb]{0.56,0.35,0.01}{\textbf{\textit{#1}}}}
\usepackage{longtable,booktabs,array}
\usepackage{calc} % for calculating minipage widths
% Correct order of tables after \paragraph or \subparagraph
\usepackage{etoolbox}
\makeatletter
\patchcmd\longtable{\par}{\if@noskipsec\mbox{}\fi\par}{}{}
\makeatother
% Allow footnotes in longtable head/foot
\IfFileExists{footnotehyper.sty}{\usepackage{footnotehyper}}{\usepackage{footnote}}
\makesavenoteenv{longtable}
\usepackage{graphicx}
\makeatletter
\def\maxwidth{\ifdim\Gin@nat@width>\linewidth\linewidth\else\Gin@nat@width\fi}
\def\maxheight{\ifdim\Gin@nat@height>\textheight\textheight\else\Gin@nat@height\fi}
\makeatother
% Scale images if necessary, so that they will not overflow the page
% margins by default, and it is still possible to overwrite the defaults
% using explicit options in \includegraphics[width, height, ...]{}
\setkeys{Gin}{width=\maxwidth,height=\maxheight,keepaspectratio}
% Set default figure placement to htbp
\makeatletter
\def\fps@figure{htbp}
\makeatother
\setlength{\emergencystretch}{3em} % prevent overfull lines
\providecommand{\tightlist}{%
  \setlength{\itemsep}{0pt}\setlength{\parskip}{0pt}}
\setcounter{secnumdepth}{5}
\usepackage{booktabs}
\ifLuaTeX
  \usepackage{selnolig}  % disable illegal ligatures
\fi
\usepackage[]{natbib}
\bibliographystyle{plainnat}
\IfFileExists{bookmark.sty}{\usepackage{bookmark}}{\usepackage{hyperref}}
\IfFileExists{xurl.sty}{\usepackage{xurl}}{} % add URL line breaks if available
\urlstyle{same} % disable monospaced font for URLs
\hypersetup{
  pdftitle={Métodos de matemáticas aplicadas},
  pdfauthor={Héctor Andrés Chang-Lara},
  hidelinks,
  pdfcreator={LaTeX via pandoc}}

\title{Métodos de matemáticas aplicadas}
\author{Héctor Andrés Chang-Lara}
\date{2022-07-16}

\usepackage{amsthm}
\newtheorem{theorem}{Theorem}[chapter]
\newtheorem{lemma}{Lemma}[chapter]
\newtheorem{corollary}{Corollary}[chapter]
\newtheorem{proposition}{Proposition}[chapter]
\newtheorem{conjecture}{Conjecture}[chapter]
\theoremstyle{definition}
\newtheorem{definition}{Definition}[chapter]
\theoremstyle{definition}
\newtheorem{example}{Example}[chapter]
\theoremstyle{definition}
\newtheorem{exercise}{Exercise}[chapter]
\theoremstyle{definition}
\newtheorem{hypothesis}{Hypothesis}[chapter]
\theoremstyle{remark}
\newtheorem*{remark}{Remark}
\newtheorem*{solution}{Solution}
\begin{document}
\maketitle

{
\setcounter{tocdepth}{1}
\tableofcontents
}
\hypertarget{intro}{%
\chapter{Intro}\label{intro}}

La siguiente figura ilustra cuatro puntos masivos unidos por tres barras de longitudes conocidas \(\ell_{01}, \ell_{12}, \ell_{23}\), y masas despreciables. Los extremos etiquetados por \(0\) y \(3\) tienen posiciones fijas y los nodos intermedios de masas \(m_1\) y \(m_2\) ocupan posiciones de equilibrio. ¿A partir de cuales ecuaciones se podrían determinar las posiciones \(q_i=(x_i,y_i)\) de estos nodos?

\begin{figure}

{\centering \includegraphics[width=0.5\linewidth]{./entramado} 

}

\caption{Entramado.}\label{fig:unnamed-chunk-1}
\end{figure}

Antes de proceder a plantear el sistema de ecuaciones recordemos que por lo general el número de ecuaciones e incógnitas deben ser iguales para que este esté bien planteado, es decir que existan soluciones y que sean únicas (al menos localmente). En nuestro caso tenemos cuatro incógnitas, los dos pares de coordenadas de cada nodo libre. Además debemos considerar las restricciones impuestas por las distancias entre los nodos, es decir tres ecuaciones. Hasta el momento el sistema es indeterminado, tiene más incógnitas (4) que ecuaciones (3), sin embargo aún nos falta incorporar la información del fenómeno de equilibrio.

En cada nodo libre actúan tres fuerzas: dos tensiones y la gravedad \((= -m_ige_y)\). Por ejemplo, la tensión \(T_{12}\) sobre el nodo 1 y que se produce sobre el segmento que une los nodos 1 y 2 es proporcional al vector \(q_2-q_1\), es decir \(T_{12} = \lambda_{12} (q_2-q_1)\) para un cierto escalar \(\lambda_{12}\). Similarmente podemos razonar sobre las demás interacciones, introduciendo así cuatro nuevas variables \(\lambda_{10}, \lambda_{12}, \lambda_{21}\), y \(\lambda_{23}\). Para que el sistema se encuentre en equilibrio, la suma de las fuerzas sobre cada nodo debe anularse, lo cual nos da igualmente cuatro ecuaciones:

\[
\begin{cases}
\lambda_{10}(x_{0}-x_{1})+\lambda_{12}(x_{2}-x_{1}) = 0\\
\lambda_{10}(y_{0}-y_{1})+\lambda_{12}(y_{2}-y_{1}) = m_{1}g\\
\lambda_{23}(x_{3}-x_{2})+\lambda_{21}(x_{1}-x_{2}) = 0\\
\lambda_{23}(y_{3}-y_{2})+\lambda_{21}(y_{1}-y_{2}) = m_{2}g
\end{cases}
\]

Pareciera que no hemos logrado mucho en términos del sistema que sigue siendo indeterminado con ocho incógnitas (2 \(x\)'s, 2 \(y\)'s y 4 \(\lambda\)'s) y siete ecuaciones (3 distancias y 4 balances de fuerzas). Sin embargo, la tercera ley de Newton nos dice que las interacciones entre pares de nodos guarda una simetría: toda acción produce una reacción opuesta de la misma magnitud. En nuestro modelo esto se refleja en \(T_{12} = -T_{21}\), de donde obtenemos la última ecuación

\[
\lambda_{12}=\lambda_{21}.
\]

De hecho es más sencillo eliminar una de las incógnitas (\(\lambda_{21}\)) que añadir otra ecuación. En conclusión obtenemos el siguiente sistema con siete ecuaciones e incógnitas

\[
    \begin{cases}
    &\lambda_{10}(x_{0}-x_{1})+\lambda_{12}(x_{2}-x_{1}) = 0\\
    &\lambda_{10}(y_{0}-y_{1})+\lambda_{12}(y_{2}-y_{1}) = m_{1}g\\
    &\lambda_{23}(x_{3}-x_{2})+\lambda_{12}(x_{1}-x_{2}) = 0\\
    &\lambda_{23}(y_{3}-y_{2})+\lambda_{12}(y_{1}-y_{2}) = m_{2}g\\
    &(x_1-x_0)^2 + (y_1-y_0)^2 = \ell_{01}^2\\
    &(x_1-x_2)^2 + (y_1-y_2)^2 = \ell_{12}^2\\
    &(x_2-x_3)^2 + (y_2-y_3)^2 = \ell_{23}^2
    \end{cases}
\]

Una forma de obtener solución a este sistema es el método de Newton. Por ejemplo, para los valores \(\ell_{01}=\sqrt{5}, \ell_{12}=\sqrt{5}, \ell_{23}=\sqrt{10}, m_1=1, m_2=2, q_0=(0,0), q_3=(4,0)\) la siguiente implementación ilustra como obtener la solución usando Python\footnote{\textbf{Advertencia:} El código es sensible a las condiciones iniciales para la iteración y no siempre converge.}.

\begin{Shaded}
\begin{Highlighting}[]
\CommentTok{\#Librerías}

\ImportTok{import}\NormalTok{ matplotlib.pyplot }\ImportTok{as}\NormalTok{ plt}
\ImportTok{import}\NormalTok{ numpy }\ImportTok{as}\NormalTok{ np}
\ImportTok{from}\NormalTok{ scipy.optimize }\ImportTok{import}\NormalTok{ fsolve}

\CommentTok{\#Parámetros}

\NormalTok{l01}\OperatorTok{=}\NormalTok{np.sqrt(}\DecValTok{5}\NormalTok{)}
\NormalTok{l12}\OperatorTok{=}\NormalTok{np.sqrt(}\DecValTok{5}\NormalTok{)}
\NormalTok{l23}\OperatorTok{=}\NormalTok{np.sqrt(}\DecValTok{10}\NormalTok{)}
\NormalTok{x3,y3}\OperatorTok{=}\DecValTok{4}\NormalTok{,}\DecValTok{0}
\NormalTok{m1}\OperatorTok{=}\DecValTok{1}
\NormalTok{m2}\OperatorTok{=}\DecValTok{2}

\CommentTok{\#Sistema de ecuaciones y gráfica}

\KeywordTok{def}\NormalTok{ f(x):}
\NormalTok{  x1,y1,x2,y2,lambda01,lambda12,lambda23 }\OperatorTok{=}\NormalTok{ x}
\NormalTok{  f}\OperatorTok{=}\NormalTok{np.zeros(}\DecValTok{7}\NormalTok{)}
\NormalTok{  f[}\DecValTok{0}\NormalTok{] }\OperatorTok{=}\NormalTok{ x1}\OperatorTok{**}\DecValTok{2}\OperatorTok{+}\NormalTok{y1}\OperatorTok{**}\DecValTok{2}\OperatorTok{{-}}\NormalTok{l01}\OperatorTok{**}\DecValTok{2}
\NormalTok{  f[}\DecValTok{1}\NormalTok{] }\OperatorTok{=}\NormalTok{ (x2}\OperatorTok{{-}}\NormalTok{x1)}\OperatorTok{**}\DecValTok{2}\OperatorTok{+}\NormalTok{(y2}\OperatorTok{{-}}\NormalTok{y1)}\OperatorTok{**}\DecValTok{2}\OperatorTok{{-}}\NormalTok{l12}\OperatorTok{**}\DecValTok{2}
\NormalTok{  f[}\DecValTok{2}\NormalTok{] }\OperatorTok{=}\NormalTok{ (x3}\OperatorTok{{-}}\NormalTok{x2)}\OperatorTok{**}\DecValTok{2}\OperatorTok{+}\NormalTok{(y3}\OperatorTok{{-}}\NormalTok{y2)}\OperatorTok{**}\DecValTok{2}\OperatorTok{{-}}\NormalTok{l23}\OperatorTok{**}\DecValTok{2}
\NormalTok{  f[}\DecValTok{3}\NormalTok{] }\OperatorTok{=} \OperatorTok{{-}}\NormalTok{lambda01}\OperatorTok{*}\NormalTok{x1}\OperatorTok{+}\NormalTok{lambda12}\OperatorTok{*}\NormalTok{(x2}\OperatorTok{{-}}\NormalTok{x1)}
\NormalTok{  f[}\DecValTok{4}\NormalTok{] }\OperatorTok{=} \OperatorTok{{-}}\NormalTok{lambda01}\OperatorTok{*}\NormalTok{y1}\OperatorTok{+}\NormalTok{lambda12}\OperatorTok{*}\NormalTok{(y2}\OperatorTok{{-}}\NormalTok{y1)}\OperatorTok{{-}}\NormalTok{m1}
\NormalTok{  f[}\DecValTok{5}\NormalTok{] }\OperatorTok{=}\NormalTok{ lambda12}\OperatorTok{*}\NormalTok{(x1}\OperatorTok{{-}}\NormalTok{x2)}\OperatorTok{+}\NormalTok{lambda23}\OperatorTok{*}\NormalTok{(x3}\OperatorTok{{-}}\NormalTok{x2)}
\NormalTok{  f[}\DecValTok{6}\NormalTok{] }\OperatorTok{=}\NormalTok{ lambda12}\OperatorTok{*}\NormalTok{(y1}\OperatorTok{{-}}\NormalTok{y2)}\OperatorTok{+}\NormalTok{lambda23}\OperatorTok{*}\NormalTok{(y3}\OperatorTok{{-}}\NormalTok{y2)}\OperatorTok{{-}}\NormalTok{m2}
  \ControlFlowTok{return}\NormalTok{ f}

\NormalTok{r }\OperatorTok{=}\NormalTok{ fsolve(f,[}\DecValTok{1}\NormalTok{,}\OperatorTok{{-}}\DecValTok{1}\NormalTok{,}\DecValTok{3}\NormalTok{,}\OperatorTok{{-}}\DecValTok{2}\NormalTok{,}\DecValTok{0}\NormalTok{,}\DecValTok{0}\NormalTok{,}\DecValTok{0}\NormalTok{])}
\NormalTok{plt.plot([}\DecValTok{0}\NormalTok{,r[}\DecValTok{0}\NormalTok{],r[}\DecValTok{2}\NormalTok{],x3], [}\DecValTok{0}\NormalTok{,r[}\DecValTok{1}\NormalTok{],r[}\DecValTok{3}\NormalTok{],y3],[}\DecValTok{0}\NormalTok{,r[}\DecValTok{0}\NormalTok{],r[}\DecValTok{2}\NormalTok{],x3], [}\DecValTok{0}\NormalTok{,r[}\DecValTok{1}\NormalTok{],r[}\DecValTok{3}\NormalTok{],y3],}\StringTok{\textquotesingle{}ro\textquotesingle{}}\NormalTok{)}
\end{Highlighting}
\end{Shaded}

\includegraphics{_main_files/figure-latex/unnamed-chunk-2-1.pdf}

Estas ideas son fácilmente generalizables a configuraciones lineales con más nodos. En el límite se obtiene el \emph{problema de la catenaria}. También podemos considerar estructuras más complejas, por ejemplo un pañuelo sujeto por las esquinas. Para poder dar una generalización de estos modelos presentamos en la siguiente sección algunas nociones básicas de teoría de grafos. Una referencia entretenida con aplicaciones en arquitectura está en el siguiente enlace:

\begin{longtable}[]{@{}
  >{\centering\arraybackslash}p{(\columnwidth - 0\tabcolsep) * \real{1.0000}}@{}}
\toprule()
\begin{minipage}[b]{\linewidth}\centering
\includegraphics{./0.jpg}
\end{minipage} \\
\midrule()
\endhead
\href{https://youtu.be/KXP_kPPc7LY}{Diseñar estructuras\ldots{} ¿sin cálculos? 🤔 La magia de la CATENARIA} \\
\bottomrule()
\end{longtable}

\hypertarget{hello-bookdown}{%
\chapter{Hello bookdown}\label{hello-bookdown}}

All chapters start with a first-level heading followed by your chapter title, like the line above. There should be only one first-level heading (\texttt{\#}) per .Rmd file.

\hypertarget{a-section}{%
\section{A section}\label{a-section}}

All chapter sections start with a second-level (\texttt{\#\#}) or higher heading followed by your section title, like the sections above and below here. You can have as many as you want within a chapter.

\hypertarget{an-unnumbered-section}{%
\subsection*{An unnumbered section}\label{an-unnumbered-section}}
\addcontentsline{toc}{subsection}{An unnumbered section}

Chapters and sections are numbered by default. To un-number a heading, add a \texttt{\{.unnumbered\}} or the shorter \texttt{\{-\}} at the end of the heading, like in this section.

\hypertarget{cross}{%
\chapter{Cross-references}\label{cross}}

Cross-references make it easier for your readers to find and link to elements in your book.

\hypertarget{chapters-and-sub-chapters}{%
\section{Chapters and sub-chapters}\label{chapters-and-sub-chapters}}

There are two steps to cross-reference any heading:

\begin{enumerate}
\def\labelenumi{\arabic{enumi}.}
\tightlist
\item
  Label the heading: \texttt{\#\ Hello\ world\ \{\#nice-label\}}.

  \begin{itemize}
  \tightlist
  \item
    Leave the label off if you like the automated heading generated based on your heading title: for example, \texttt{\#\ Hello\ world} = \texttt{\#\ Hello\ world\ \{\#hello-world\}}.
  \item
    To label an un-numbered heading, use: \texttt{\#\ Hello\ world\ \{-\#nice-label\}} or \texttt{\{\#\ Hello\ world\ .unnumbered\}}.
  \end{itemize}
\item
  Next, reference the labeled heading anywhere in the text using \texttt{\textbackslash{}@ref(nice-label)}; for example, please see Chapter \ref{cross}.

  \begin{itemize}
  \tightlist
  \item
    If you prefer text as the link instead of a numbered reference use: \protect\hyperlink{cross}{any text you want can go here}.
  \end{itemize}
\end{enumerate}

\hypertarget{captioned-figures-and-tables}{%
\section{Captioned figures and tables}\label{captioned-figures-and-tables}}

Figures and tables \emph{with captions} can also be cross-referenced from elsewhere in your book using \texttt{\textbackslash{}@ref(fig:chunk-label)} and \texttt{\textbackslash{}@ref(tab:chunk-label)}, respectively.

See Figure \ref{fig:nice-fig}.

\begin{Shaded}
\begin{Highlighting}[]
\FunctionTok{par}\NormalTok{(}\AttributeTok{mar =} \FunctionTok{c}\NormalTok{(}\DecValTok{4}\NormalTok{, }\DecValTok{4}\NormalTok{, .}\DecValTok{1}\NormalTok{, .}\DecValTok{1}\NormalTok{))}
\FunctionTok{plot}\NormalTok{(pressure, }\AttributeTok{type =} \StringTok{\textquotesingle{}b\textquotesingle{}}\NormalTok{, }\AttributeTok{pch =} \DecValTok{19}\NormalTok{)}
\end{Highlighting}
\end{Shaded}

\begin{figure}

{\centering \includegraphics[width=0.8\linewidth]{_main_files/figure-latex/nice-fig-3} 

}

\caption{Here is a nice figure!}\label{fig:nice-fig}
\end{figure}

Don't miss Table \ref{tab:nice-tab}.

\begin{Shaded}
\begin{Highlighting}[]
\NormalTok{knitr}\SpecialCharTok{::}\FunctionTok{kable}\NormalTok{(}
  \FunctionTok{head}\NormalTok{(pressure, }\DecValTok{10}\NormalTok{), }\AttributeTok{caption =} \StringTok{\textquotesingle{}Here is a nice table!\textquotesingle{}}\NormalTok{,}
  \AttributeTok{booktabs =} \ConstantTok{TRUE}
\NormalTok{)}
\end{Highlighting}
\end{Shaded}

\begin{table}

\caption{\label{tab:nice-tab}Here is a nice table!}
\centering
\begin{tabular}[t]{rr}
\toprule
temperature & pressure\\
\midrule
0 & 0.0002\\
20 & 0.0012\\
40 & 0.0060\\
60 & 0.0300\\
80 & 0.0900\\
\addlinespace
100 & 0.2700\\
120 & 0.7500\\
140 & 1.8500\\
160 & 4.2000\\
180 & 8.8000\\
\bottomrule
\end{tabular}
\end{table}

\hypertarget{parts}{%
\chapter{Parts}\label{parts}}

You can add parts to organize one or more book chapters together. Parts can be inserted at the top of an .Rmd file, before the first-level chapter heading in that same file.

Add a numbered part: \texttt{\#\ (PART)\ Act\ one\ \{-\}} (followed by \texttt{\#\ A\ chapter})

Add an unnumbered part: \texttt{\#\ (PART\textbackslash{}*)\ Act\ one\ \{-\}} (followed by \texttt{\#\ A\ chapter})

Add an appendix as a special kind of un-numbered part: \texttt{\#\ (APPENDIX)\ Other\ stuff\ \{-\}} (followed by \texttt{\#\ A\ chapter}). Chapters in an appendix are prepended with letters instead of numbers.

\hypertarget{footnotes-and-citations}{%
\chapter{Footnotes and citations}\label{footnotes-and-citations}}

\hypertarget{footnotes}{%
\section{Footnotes}\label{footnotes}}

Footnotes are put inside the square brackets after a caret \texttt{\^{}{[}{]}}. Like this one \footnote{This is a footnote.}.

\hypertarget{citations}{%
\section{Citations}\label{citations}}

Reference items in your bibliography file(s) using \texttt{@key}.

For example, we are using the \textbf{bookdown} package \citep{R-bookdown} (check out the last code chunk in index.Rmd to see how this citation key was added) in this sample book, which was built on top of R Markdown and \textbf{knitr} \citep{xie2015} (this citation was added manually in an external file book.bib).
Note that the \texttt{.bib} files need to be listed in the index.Rmd with the YAML \texttt{bibliography} key.

The RStudio Visual Markdown Editor can also make it easier to insert citations: \url{https://rstudio.github.io/visual-markdown-editing/\#/citations}

\hypertarget{blocks}{%
\chapter{Blocks}\label{blocks}}

\hypertarget{equations}{%
\section{Equations}\label{equations}}

Here is an equation.

\begin{equation} 
  f\left(k\right) = \binom{n}{k} p^k\left(1-p\right)^{n-k}
  \label{eq:binom}
\end{equation}

You may refer to using \texttt{\textbackslash{}@ref(eq:binom)}, like see Equation \eqref{eq:binom}.

\hypertarget{theorems-and-proofs}{%
\section{Theorems and proofs}\label{theorems-and-proofs}}

Labeled theorems can be referenced in text using \texttt{\textbackslash{}@ref(thm:tri)}, for example, check out this smart theorem \ref{thm:tri}.

\begin{theorem}
\protect\hypertarget{thm:tri}{}\label{thm:tri}For a right triangle, if \(c\) denotes the \emph{length} of the hypotenuse
and \(a\) and \(b\) denote the lengths of the \textbf{other} two sides, we have
\[a^2 + b^2 = c^2\]
\end{theorem}

Read more here \url{https://bookdown.org/yihui/bookdown/markdown-extensions-by-bookdown.html}.

\hypertarget{callout-blocks}{%
\section{Callout blocks}\label{callout-blocks}}

The R Markdown Cookbook provides more help on how to use custom blocks to design your own callouts: \url{https://bookdown.org/yihui/rmarkdown-cookbook/custom-blocks.html}

\hypertarget{sharing-your-book}{%
\chapter{Sharing your book}\label{sharing-your-book}}

\hypertarget{publishing}{%
\section{Publishing}\label{publishing}}

HTML books can be published online, see: \url{https://bookdown.org/yihui/bookdown/publishing.html}

\hypertarget{pages}{%
\section{404 pages}\label{pages}}

By default, users will be directed to a 404 page if they try to access a webpage that cannot be found. If you'd like to customize your 404 page instead of using the default, you may add either a \texttt{\_404.Rmd} or \texttt{\_404.md} file to your project root and use code and/or Markdown syntax.

\hypertarget{metadata-for-sharing}{%
\section{Metadata for sharing}\label{metadata-for-sharing}}

Bookdown HTML books will provide HTML metadata for social sharing on platforms like Twitter, Facebook, and LinkedIn, using information you provide in the \texttt{index.Rmd} YAML. To setup, set the \texttt{url} for your book and the path to your \texttt{cover-image} file. Your book's \texttt{title} and \texttt{description} are also used.

This \texttt{gitbook} uses the same social sharing data across all chapters in your book- all links shared will look the same.

Specify your book's source repository on GitHub using the \texttt{edit} key under the configuration options in the \texttt{\_output.yml} file, which allows users to suggest an edit by linking to a chapter's source file.

Read more about the features of this output format here:

\url{https://pkgs.rstudio.com/bookdown/reference/gitbook.html}

Or use:

\begin{Shaded}
\begin{Highlighting}[]
\NormalTok{?bookdown}\SpecialCharTok{::}\NormalTok{gitbook}
\end{Highlighting}
\end{Shaded}


  \bibliography{book.bib,packages.bib}

\end{document}
